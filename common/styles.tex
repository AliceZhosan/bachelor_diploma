% !TEX encoding   = UTF8
% !TEX spellcheck = en_US

%%% Template
\DeclareRobustCommand{\todo}{\textcolor{red}}

\AtBeginDocument{%
  \setlength{\parindent}{2.5em}
}


%%% Captions
\setlength{\abovecaptionskip}{0pt}
\setlength{\belowcaptionskip}{0pt}
\captionwidth{\linewidth}
\normalcaptionwidth


%%% Tables
\ifnumequal{\value{tabcap}}{0}{%
  \newcommand{\tabcapalign}{\raggedright}   % по левому краю страницы или аналога parbox
  \renewcommand{\tablabelsep}{~\cyrdash\ }  % тире как разделитель идентификатора с номером от наименования
  \newcommand{\tabtitalign}{}
}{%
  \ifnumequal{\value{tablaba}}{0}{%
    \newcommand{\tabcapalign}{\raggedright}      % по левому краю страницы или аналога parbox
  }{}

  \ifnumequal{\value{tablaba}}{1}{%
    \newcommand{\tabcapalign}{\centering}        % по центру страницы или аналога parbox
  }{}

  \ifnumequal{\value{tablaba}}{2}{%
    \newcommand{\tabcapalign}{\raggedleft}       % по правому краю страницы или аналога parbox
  }{}

  \ifnumequal{\value{tabtita}}{0}{%
    \newcommand{\tabtitalign}{\par\raggedright}  % по левому краю страницы или аналога parbox
  }{}

  \ifnumequal{\value{tabtita}}{1}{%
    \newcommand{\tabtitalign}{\par\centering}    % по центру страницы или аналога parbox
  }{}

  \ifnumequal{\value{tabtita}}{2}{%
    \newcommand{\tabtitalign}{\par\raggedleft}   % по правому краю страницы или аналога parbox
  }{}
}

\precaption{\tabcapalign}                        % всегда идет перед подписью или \legend
\captionnamefont{\normalfont\normalsize}         % Шрифт надписи «Таблица #»; также определяет шрифт у \legend
\captiondelim{\tablabelsep}                      % разделитель идентификатора с номером от наименования
\captionstyle[\tabtitalign]{\tabtitalign}
\captiontitlefont{\normalfont\normalsize}        % Шрифт с текстом подписи


%%% Figures
\setfloatadjustment{figure}{%
  \setlength{\abovecaptionskip}{0pt}
  \setlength{\belowcaptionskip}{0pt}
  \precaption{}
  \captionnamefont{\normalfont\normalsize}
  \captiondelim{\figlabelsep}
  \captionstyle[\centering]{\centering}
  \captiontitlefont{\normalfont\normalsize}
  \postcaption{}
}


%%% Subfigures captions
\newsubfloat{figure}
\renewcommand{\thesubfigure}{\asbuk{subfigure}}
\subcaptionsize{\normalsize}
\subcaptionlabelfont{\normalfont}
\subcaptionfont{\!\!) \normalfont}  % round bracket after a letter
\subcaptionstyle{\centering}
%\subcaptionsize{\fontsize{12pt}{13pt}\selectfont} % объявляем шрифт 12pt для использования в подписях, тут же надо интерлиньяж объявлять, если не наследуется


%%% Hyper references settings
\ifluatex
  \hypersetup{
    unicode,                    % unicode encoded PDF strings
  }
\fi

\hypersetup{
  linktocpage=true,             % ссылки с номера страницы в оглавлении, списке таблиц и списке рисунков
%  linktoc=all,                  % both the section and page part are links
%  pdfpagelabels=false,          % set PDF page labels (true|false)
  plainpages=false,             % forces page anchors to be named by the Arabic form  of the page number, rather than the formatted form
  colorlinks,                   % ссылки отображаются раскрашенным текстом, а не раскрашенным прямоугольником, вокруг текста
  linkcolor={linkcolor},        % цвет ссылок типа ref, eqref и подобных
  citecolor={citecolor},        % цвет ссылок-цитат
  urlcolor={urlcolor},          % цвет гиперссылок
%  hidelinks,                    % hide links (removing color and border)
  pdftitle={\thesisTitle},      % заголовок
  pdfauthor={\thesisAuthor},    % автор
%  pdfsubject={\thesisSubject},  % тема
  pdfkeywords={\keywords},      % ключевые слова
  pdflang={ru},
}



%%% Lists
%% Используем короткое тире (endash) для ненумерованных списков (ГОСТ 2.105-95, пункт 4.1.7, требует дефиса, но так лучше смотрится)
\renewcommand{\labelitemi}{\normalfont\bfseries{--}}

%% Перечисление строчными буквами русского алфавита (ГОСТ 2.105-95, 4.1.7)
\makeatletter
\AddEnumerateCounter{\Asbuk}{\russian@Alph}{Щ}
\AddEnumerateCounter{\asbuk}{\russian@alph}{щ}  % Управляем списками/перечислениями через пакет enumitem, а он 'не знает' про asbuk, потому 'учим' его
\makeatother
%\renewcommand{\theenumi}{\asbuk{enumi}}      % первый уровень нумерации
%\renewcommand{\labelenumi}{\theenumi)}       % первый уровень нумерации
\renewcommand{\theenumii}{\asbuk{enumii}}     % второй уровень нумерации
\renewcommand{\labelenumii}{\theenumii)}      % второй уровень нумерации
\renewcommand{\theenumiii}{\arabic{enumiii}}  % третий уровень нумерации
\renewcommand{\labelenumiii}{\theenumiii)}    % третий уровень нумерации

\setlist{nosep,%                              % единый стиль для всех списков (пакет enumitem), без дополнительных интервалов.
  labelindent=\parindent,leftmargin=*%        % каждый пункт, подпункт и перечисление записывают с абзацного отступа (ГОСТ 2.105-95, 4.1.8)
}

%%% Code
\DeclareRobustCommand*{\name}{\texttt}
\DeclareRobustCommand*{\code}[1]{\name{\small #1}}
\DeclareRobustCommand*{\lang}[1]{\name{#1}}
