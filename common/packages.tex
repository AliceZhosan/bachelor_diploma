% !TEX encoding   = UTF8
% !TEX spellcheck = en_US

\RequirePackage{etoolbox}[2015/08/02]  % Advanced checking of different conditions
\providebool{presentation}


%%% Page layout
\usepackage{pdflscape}
\usepackage{geometry}


%%% Mathematics
\usepackage{amsthm,amsfonts,amsmath,amscd}
\usepackage{mathtools}             % Add environment 'multlined'
\usepackage{dsfont}
%\usepackage{mathspec}              % Do not work with LuaLaTeX
\usepackage{unicode-math}


%%% Encodings and fonts
%% Settings for 14 pt font size %%
%% Формирование переменных и констант для сравнения (один раз для всех подключаемых файлов)
%% должно располагаться до вызова пакета fontspec или polyglossia, потому что они сбивают его работу
\newlength{\curtextsize}
\newlength{\bigtextsize}
\setlength{\bigtextsize}{13.9pt}

\makeatletter
%\show\f@size                      % неплохо для отслеживания, но вызывает стопорение процесса, если документ компилируется без команды  -interaction=nonstopmode
\setlength{\curtextsize}{\f@size pt}
\makeatother

\usepackage{polyglossia}           % Automatically load 'fontspec'


%%% Paragraph layout
\usepackage{indentfirst}


%%% Colors
\ifpresentation
\else
  \usepackage[svgnames,table,hyperref]{xcolor} %,cmyk
\fi


%%% Tables
\usepackage{longtable, ltcaption}
\usepackage{multirow, makecell}    % Advanced formatting


%%% General layout
\usepackage{soulutf8}              % Underlying with hyphenation
\usepackage{icomma}


%%% Оптимизация расстановки переносов и длины последней строки абзаца
\IfFileExists{impnattypo.sty}{% проверка установленности пакета impnattypo
  \ifluatex
    \ifnumequal{\value{draft}}{1}{% Черновик
      \usepackage[hyphenation, lastparline, nosingleletter, homeoarchy, rivers, draft]{impnattypo}
    }{% Чистовик
      \usepackage[hyphenation, lastparline, nosingleletter]{impnattypo}
    }
  \else
    \usepackage[hyphenation, lastparline]{impnattypo}
  \fi
}{}


%%% Hyper references
\usepackage{hyperref}[2012/11/06]


%%% Figures
\usepackage{graphicx}[2014/04/25]


%%% Counters
\usepackage[figure,table]{totalcount}  % Счётчик рисунков и таблиц
\usepackage{totcount}                  % Пакет создания счётчиков на основе последнего номера подсчитываемого элемента (может требовать дважды компилировать документ)
\usepackage{totpages}                  % Счётчик страниц, совместимый с hyperref (ссылается на номер последней страницы). Желательно ставить последним пакетом в преамбуле


%%% Smart references
%\ifpresentation
%\else
%  \usepackage{cleveref}
%
%  \creflabelformat{equation}{#2#1#3}     % Формат по умолчанию ставил круглые скобки вокруг каждого номера ссылки, теперь просто номера ссылок без какого-либо дополнительного оформления
%  \crefrangelabelformat{equation}{#3#1#4\cyrdash#5#2#6}  % Интервалы в русском языке принято делать через тире, если иное не оговорено
%
%  % решение проблемы с "и" в \labelcref
%  % https://tex.stackexchange.com/a/455124/104425
%  \DeclareTextSymbol{\cyri}\UnicodeEncodingName{"0438} % и
%
%  % Добавление возможности использования пробелов в \labelcref
%  % https://tex.stackexchange.com/a/340502/104425
%  \usepackage{kvsetkeys}
%  \makeatletter
%  \let\org@@cref\@cref
%  \renewcommand*{\@cref}[2]{%
%    \edef\process@me{%
%      \noexpand\org@@cref{#1}{\zap@space#2 \@empty}%
%    }\process@me
%  }
%  \makeatother
%\fi

\ifnumequal{\value{draft}}{1}{% Черновик
  \usepackage[firstpage]{draftwatermark}
  \SetWatermarkText{DRAFT}
  \SetWatermarkFontSize{14pt}
  \SetWatermarkScale{15}
  \SetWatermarkAngle{45}
}{}


%%% Исправление положения якорей подписей (под)рисунков
% Без hypcap и патча, при клике по ссылке на подрисунок, просмотрщик pdf прыгает "к подписи" а не "к рисунку".
% Подробнее: https://github.com/AndreyAkinshin/Russian-Phd-LaTeX-Dissertation-Template/issues/238
% (!) Даже с патчем, если мешать в одной фиге разные типы подфиг (subbottom и subcaption) - ссылки всё равно будут работать неправильно  (см. https://www.overleaf.com/read/czmbmmtnqrrg ).
\ifpresentation
\else
  \usepackage[all]{hypcap}

  \makeatletter
  \ltx@ifclasslater{memoir}{2018/12/13}{
    % Предполагается, что в следующей версии класс будет исправлен
    \typeout{Assuming this version of memoir is free from the jumping-to-caption bug.}
  }{
    \RequirePackage{xpatch}

    \newcommand\mem@step@subcounter{\refstepcounter{sub\@captype}\@contkeep}

    \xpatchcmd{\@memsubbody}%
    {\refstepcounter{sub\@captype}\@contkeep}% search pattern
    {}% replacement
    {\typeout{@memsubbody is patched}}%
    {\typeout{@memsubbody is NOT patched}}%

    \xpatchcmd{\@memcontsubbody}%
    {\refstepcounter{sub\@captype}\@contkeep}% pattern
    {}% replacement
    {\typeout{@memcontsubbody is patched}}%
    {\typeout{@memcontsubbody is NOT patched}}%

    \xpatchcmd{\@memsubfloat}%
    {\vbox\bgroup}% search pattern
    {\vbox\bgroup\mem@step@subcounter}% replacement
    {\typeout{@memsubfloat patch is ok}}%
    {\typeout{@memsubfloat patch is NOT ok}}%

    \xpatchcmd{\subcaption}%
    {\refstepcounter{sub\@captype}}% search pattern
    {\H@refstepcounter{sub\@captype}}% replacement
    {\typeout{subcaption second patch is ok}}%
    {\typeout{subcaption second patch is NOT ok}}%
  }
  \makeatother
\fi


%%% Vector graphics
\usepackage{tikz}
\usetikzlibrary{arrows,decorations.pathmorphing,backgrounds,positioning,fit,calc}
\usetikzlibrary{arrows.meta}
\usetikzlibrary{shapes,shapes.misc}
\usetikzlibrary{graphs, graphdrawing}
\usegdlibrary{trees, layered}

% для уменьшения размера графики tikz
\usepackage{adjustbox}

% если текущий процесс запущен библиотекой tikz-external, то предкомпиляция должна быть включена
\ifdefined\tikzexternalrealjob
  \setcounter{imgprecompile}{1}
\fi

\ifnumequal{\value{imgprecompile}}{1}{% Только если у нас включена предкомпиляция
  \usetikzlibrary{external}              % подключение возможности предкомпиляции
  \tikzexternalize[prefix=images/cache/] % activate! % здесь можно указать отдельную папку для скомпилированных файлов
  \ifxetex
    \tikzset{external/up to date check={diff}}
  \fi
}{}