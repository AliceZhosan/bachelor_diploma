% !TEX encoding   = UTF8
% !TEX spellcheck = ru_RU

%%================
\Chapter{Введение}
%%================

Рост мощности и доступности современных вычислительных ресурсов позволяет решать более трудоёмкие задачи. Моделирование крупных вихрей. Оптимизационные задачи, где необходимо перестроение геометрии частей летательного аппарата и, соответственно, расчётной сетки. Моделирование с учётом нескольких дисциплин, например, аэродинамика плюс прочность, аэродинамика плюс химические реакции (горение).

В ЦАГИ уже несколько лет активно ведётся разработка кода ZOOM для решения научных и промышленных задач аэродинамики на неструктурированных сетках. Одной из веток является семейство схем высокого порядка точности на базе метода Галёркина с разрывными базисными функциями (DG). В коде ZOOM DG реализованы уравнения Навье-Стокса, осреднённые по Рейнольдсу и замыкаемые моделью турбулентности Спаларта-Альмараса, а также вихреразрешающий метод DDES для расчёта нестационарных течений. Полностью реализована поддержка только гексаэдральных элементов.

В долгосрочной перспективе для использования гибридных неструктурированных сеток необходимо добавить тетраэдры, призмы, которые часто появляются в пограничных слоях, и пирамиды как промежуточное звено. Поддерживаются только такие сетки, где ячейки стыкуются грань-в-грань, поэтому нельзя напрямую перейти от призм и гексаэдров сразу к тетраэдрам без включения пирамид.

\textbf{Целью данной работы} является разработка и реализация поддержки тетраэдральных элементов в солвере ZOOM DG.

\textbf{Выбор темы} обусловлен её \textbf{практической значимостью}, которая заключается в расширении функциональных возможностей расчётного модуля ZOOM DG.

\textbf{Актуальность темы} определяется применением схем высокого порядка точности и преимуществами неструктурированных расчётных сеток:
\begin{enumerate}
	\item Большая степень автоматизации при построении, в то время как структурированная сетка требует трудоёмкой ручной работы. Это в разы ускоряет процесс подготовки расчёта.
	\item Эффективность использования многопроцессорных вычислительных систем за счёт возможности хорошо распределить нагрузку между процессорами и уменьшить площадь поверхности для обменов.
\end{enumerate}

\textbf{Достоверность} полученных результатов подтверждается использованием аналитических решений для проверки преобразования при переходе в локальную систему координат элементов сетки и вычисления поверхностных и объёмных интегралов. Работа численного метода проверяется на простой задаче дозвукового невязкого обтекания цилиндра, решение которой известно.



%%========================
\Chapter{Обзор литературы}
%%========================

В работах группы Басси~\cite{Tesini:2008:en, Bassi:1997:en} изложены алгоритмы ортонормирования базисных функций, способ сокращения вычислительных затрат для интегрирования и способ вычисления градиентов независимых переменных. Эти работы, а также опыт Волкова \cite{VolkovA:2010:ru}, легли в основу кода ZOOM DG \cite{Podaruev:2017}.

В работе Тесини \cite{Tesini:2008:en} описан метод DG в двумерной постановке, гауссовы кубатурные правила, а также изложена идея получения гауссовых кубатурных правил для тетраэдров.

Преобразования гексаэдров и тетраэдров, которые используют информацию о положении вершин и центров ребер элементов, описаны в~\cite{Zienkiewicz:2000:en}. Анализ степеней полиномов в линейных и квадратичных преобразованиях гексаэдров проведен в работе~\cite{Efremova:2018}. 

Таблицы оптимизированных правил интегрирования для разных областей находятся на веб-сайте~\cite{CubatureRules}.

В статье~\cite{SukumarCubatureRules:2020:en} описан алгоритм поиска кубатурных правил для произвольных элементов. В частности, получены оптимизированные кубатурные правила для тетраэдра вплоть до 20 порядка, из которых правила с 16 до 20 порядков являются новыми. Результаты этой работы могут быть в дальнейшем использованы для уменьшения затрат при вычислении интегралов.