% !TEX encoding   = UTF8
% !TEX spellcheck = ru_RU

%%==================
\Chapter{Заключение}
%%==================

%% Согласно ГОСТ Р 7.0.11-2011:
%% 5.3.3 В заключении диссертации излагают итоги выполненного исследования, рекомендации, перспективы дальнейшей разработки темы.
%% 9.2.3 В заключении автореферата диссертации излагают итоги данного исследования, рекомендации и перспективы дальнейшей разработки темы.
%% Поэтому имеет смысл сделать эту часть общей и загрузить из одного файла в автореферат и в диссертацию:

Основным результатом работы является внедрение поддержки тетраэдральных элементов в расчётный модуль ZOOM DG.

В ходе работы изучены правила преобразований координат для тетраэдров и гексаэдров, а также правила интегрирования по ним. Работа численного метода проверена на примере задачи дозвукового обтекания цилиндра. Результаты тестирования на грубой сетке показывают, что криволинейные элементы приводят к более точному описанию решения, а увеличение максимальной степени полиномов базисных функций на единицу "--- к уменьшению ошибки в определении коэффициента сопротивления цилиндра примерно на порядок.

В продолжение работы необходимо выполнить исследование сеточной сходимости и оценить порядок аппроксимации схем. Добавить вывод полей с подсеточным разрешением. И перейти к реализации поддержки других элементов: призм и пирамид.
