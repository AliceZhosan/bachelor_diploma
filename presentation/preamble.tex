% !TEX encoding   = UTF8
% !TEX spellcheck = ru_RU

%%========
\maketitle
%%========


\begin{frame}{Цель и задачи}
  \begin{block}{Цель работы:}\begin{itemize}
      \item[] Найти решение проблемы параллельного чтения--записи многоблочных данных
  \end{itemize}\end{block}

%  \textit{[Tesini Pietro, 2008; Волков Андрей Викторович, 2010; \ldots]}

  \bigskip
  \begin{block}{Задачи:}
    \begin{enumerate}
      \item Изучение аппаратных и программных средств
      \begin{itemize}
      	\item Параллельная файловая система Lustre \todo{[Nov 14 2021]}
      	\item Сравнение с обычной файловой системой \todo{[Nov 21 2021]}
        \item Средства MPI для работы с параллельным вводом--выводом
        \item Выбор формата файла
      \end{itemize}
    	\item Реализация в коде ZEUS \todo{[Feb 2022]}
    	\item Тестирование. Эффективность \todo{[Mar 2022]}
    \end{enumerate}
  \end{block}
\end{frame}

\note{
	В ЦАГИ уже несколько лет активно ведётся разработка кода ZOOM для решения научных и промышленных задач аэродинамики на неструктурированных сетках. Одной из веток является семейство схем высокого порядка точности на базе метода Галёркина с разрывными базисными функциями (DG)
	
  Целью работы является разработка и реализация поддержки тетраэдральных элементов в солвере ZOOM DG.
	
	Актуальность темы определяется применением схем более высокого порядка и преимуществами неструктурированных сеток: б\'ольшая степень их автоматизации и их эффективность использования в многопроцессорных вычислительных системах.
	
	Для достижения поставленной цели, исходная задача была разбита на несколько подзадач: изучение математического аппарата, реализация в коде ZOOM и тестирование написанного кода на примере решения уравнений Эйлера для задачи обтекания цилиндра.
}