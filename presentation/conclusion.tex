% !TEX encoding   = UTF8
% !TEX spellcheck = ru_RU

\begin{frame}{Заключение}
	\begin{block}{Выводы:}
		\begin{itemize}
			\item Криволинейные элементы приводят к более точному описанию решения
			\item Увеличение максимальной степени полиномов базисных функций на единицу приводит к уменьшению ошибки в определении сопротивления цилиндра примерно на порядок
		\end{itemize}
	\end{block}
	Основной результат работы "--- внедрение поддержки тетраэдральных элементов в солвер ZOOM DG
  \begin{block}{Дальше...}
  	\begin{itemize}
  		\item Исследование порядка сходимости
  		\item Доработка кода для внедрения в основную ветку (с гексаэдрами)
  		\item Внедрение призм и пирамид
  	\end{itemize}
  \end{block}
\end{frame}

\note{
	Результаты тестирования на грубой сетке показывают, что криволинейные элементы приводят к более точному описанию решения при высоких степенях базисных функций, а увеличение максимальной степени полиномов базисных функций на единицу "--- к уменьшению ошибки в определении коэффициента сопротивления цилиндра примерно на порядок. 
	
	В ходе работы изучены правила преобразований координат для тетраэдров и гексаэдров, а также правила интегрирования по ним. Работа численного метода проверена на примере задачи дозвукового обтекания цилиндра. 
	
	Основным результатом работы является внедрение поддержки тетраэдральных элементов в расчётный модуль ZOOM DG.
	
	В продолжение работы необходимо выполнить исследование сеточной сходимости и оценить порядок аппроксимации схем. Доработать код для возможности использования и тетраэдров, и гексаэдров. И перейти к реализации поддержки других элементов: призм и пирамид.
}