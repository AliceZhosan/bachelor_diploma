% !TEX encoding   = UTF8
% !TEX spellcheck = ru_RU

\begin{frame}{Обзор литературы}
	\textit{Pietro Tesini, ``An h-Multigrid Approach for High-Order Discontinuous Galerkin Methods'', 2008 г.}
	\begin{itemize}
		\item Метод Галёркина (DG)
		\item Кубатурные правила
	\end{itemize}
	\textit{Zienkiewicz, Taylor, ``The Finite Element Method'', 2000 г.}
	\begin{itemize}
		\item Преобразования элементов
	\end{itemize}
	\textit{Подаруев В.Ю. ``Метод высокого порядка точности для расчёта на суперкомпьютере характеристик турбулентных струй, истекающих из сопл гражданских самолётов'', 2017 г.}
	\begin{itemize}
		\item Подробное описание метода Галёркина (DG)
		\item Особенности реализации кода ZOOM DG
	\end{itemize}
\end{frame}

\note{
Я начал с изучения литературы. Наиболее значимые источники перечислены на слайде. В работе Тесини я познакомился с методом DG в двумерной постановке, с гауссовыми кубатурными правилами, а также с идеей получения гауссовых кубатурных правил для тетраэдров. Работы в этой группе, а также опыт Андрея Викторовича Волкова легли в основу кода ZOOM DG. В работе Зинковича и Тейлора, ``Метод конечных элементов'', я изучил преобразованиями элементов. Наконец, в работе Владимира Юрьевича Подаруева подробно описан метод Галёркина, а также описаны некоторые особенности реализации кода ZOOM DG; \alert{В этом разделе я расскажу про математический апарат.}
}