% !TEX encoding   = UTF8
% !TEX spellcheck = ru_RU

\documentclass[xcolor={svgnames,table,hyperref},9pt]{beamer}

\usefonttheme{professionalfonts}

%\usepackage{setspace}

\usepackage{amsthm,amsfonts,amsmath,amscd}
\usepackage{mathtools}     % Add environment 'multlined'
\usepackage{dsfont}
%\usepackage{mathspec}     % do not work with LuaLaTeX
\usepackage{unicode-math}

\usepackage{polyglossia}   % Automatically load 'fontspec'

\usepackage{appendixnumberbeamer}

\usepackage{minted}

\usepackage{hyperref}

\usepackage{graphicx}
\usepackage{tikz}
\usetikzlibrary{arrows,decorations.pathmorphing,backgrounds,positioning,fit,calc}
\usetikzlibrary{arrows.meta,matrix}
\usetikzlibrary{shapes,shapes.misc}


%%% Hyper references colors %%%

\definecolor{linkcolor}{rgb}{0.9,0,0}
\definecolor{citecolor}{rgb}{0,0.6,0}
\definecolor{urlcolor}{rgb}{0,0,1}


\hypersetup{
  unicode,
  linktocpage=true,
%  linktoc=all,                % both the section and page part are links
%  pdfpagelabels=false,
  plainpages=false,
  colorlinks,
  linkcolor={linkcolor},
  citecolor={citecolor},
  urlcolor={urlcolor},
%  hidelinks,
  pdflang={ru},
}


\setmainlanguage[babelshorthands=true]{russian}
\setotherlanguage{english}

\setmonofont{Fira Code}
\newfontfamily\cyrillicfonttt{Fira Code}

\defaultfontfeatures{Ligatures=TeX}  % NB! monofont settings should be before this

\usetheme{metropolis}
\setbeamercolor{normal text}{bg=white}
%\setbeamercolor{frametitle}{% for printing
%  use=normal text,
%  parent=normal text
%}
%\setbeameroption{show notes}

\metroset{progressbar=frametitle,subsectionpage=none}

\setsansfont{Fira Sans}
\newfontfamily\cyrillicfontsf{Fira Sans}

\setmathfont{STIX Two Math}
\newfontfamily\cyrillicfontmf{STIX Two Math}


\graphicspath{{images/}}


\renewcommand{\theFancyVerbLine}{%
  \textcolor{gray}{\tiny\arabic{FancyVerbLine}}%
}

\newmintinline{cpp}{fontsize=\small}
\newmintinline{text}{fontsize=\small}

\newmint{cpp}{fontsize=\small}
\newmint[txt]{text}{fontsize=\small}
\newmint{console}{fontsize=\small}

\newminted{cpp}{fontsize=\small, numbersep=0.2em}
\newminted{text}{fontsize=\small, tabsize=2}
\newminted{console}{fontsize=\small, tabsize=2}

\newmintedfile{cpp}{linenos, fontsize=\small, numbersep=0.2em}
\newmintedfile{text}{fontsize=\small, tabsize=2}
\newmintedfile{console}{fontsize=\small, tabsize=2}

\newcommand*{\code}[1]{\texttt{\small #1}}



\title
{
  Модификации компонентов ZEUS,\\
  \hfil направленные на ускорение\ldots\\[1em]
  {\small\itshape (\TeX{}нический семинар)}
}

\author
{
  В.Ю.\,Подаруев
}

\institute[ЦАГИ]
{
  Отделение аэродинамики cиловых установок, ЦАГИ, г.\,Жуковский
}

\date{\itshape 26 сентября 2019\,г.}




\begin{document}

%%========
\maketitle
%%========

\begin{frame}{Краткий план}
  \begin{block}{}
    \begin{itemize}
      \item \code{z\_make\_connection}
      \item \code{z\_dwall}
      \item Прочее\ldots
      \item Стиль кода
      \item Масштабируемость
      \item Декомпозиция
    \end{itemize}
  \end{block}
\end{frame}



%%==============================================================
\begin{frame}{Стыковка блоков: \texttt{z\_make\_connection}}
  \begin{block}{Типичное использование}
    \console`$ z_make_connection -v proj boco`
  \end{block}

  \begin{block}{Изменения}\begin{itemize}
    \item алгоритм переписан с сохранением логики;
    \item при выводе $\Delta_\text{max}$ везде используется расстояние, а не его квадрат;
    \item многопоточная реализация через \alert{\code{OpenMP}};
    \item ключик \code{ -n N } задаёт количество нитей;
    \item \alert{по умолчанию} \code{ N = число ядер CPU}.
  \end{itemize}\end{block}

  \begin{block}{Проблемы}\begin{itemize}
    \item \textcolor{ForestGreen}{[fix]} стыковка с периодическими гран. условиями;
    \item \alert{исключения} не могут покидать [обычным образом] многопоточную область кода --- это приводит к аварийному завершению;
    \item{} <<зашитые>> в коде константы, связанные с точностью.
  \end{itemize}\end{block}
\end{frame}



%%==============================================================
\begin{frame}[fragile]{Расстояние до стенки: \texttt{z\_dwall}}
  \begin{block}{Типичное использование}
    \console`$ mpiexec --hostfile hosts z_dwall -n Ncpus -v proj boco`

    где файл \code{hosts} имеет вид:

    \begin{textcode}
192.168.103.187 slots=1
192.168.103.188 slots=1
...
    \end{textcode}
  \end{block}

  \begin{block}{Изменения}\begin{itemize}
    \item многопоточная реализация через \alert{\code{OpenMP}};
    \item ключик \code{ -n N } задаёт количество нитей;
    \item \alert{по умолчанию} \code{ N = 1 } (совместимость с прежней версией).
  \end{itemize}\end{block}
  
  \begin{block}{Рекомендации}\begin{itemize}
    \item Используйте \alert{многопоточную} версию \alert{в рамках} каждого \alert{узла}. Для компоновки WBLR, $\approx$50 млн. ячеек, 10 узлов Карлсон-2:
    \txt/MPI = 1192 c vs. MPI+OpenMP = 585 с/
  \end{itemize}\end{block}
\end{frame}



%%==============================================================
\begin{frame}{Прочее\ldots}
  \begin{block}{Буферизация ввода--вывода}\begin{itemize}
    \item Системные вызовы \code{read/write} заменены на буферизованные версии из стандартной библиотеки C \code{fread/fwrite}.
  \end{itemize}\end{block}

  \begin{block}{Группировка сообщений о коррекциях}\begin{itemize}
    \item При выводе сообщения о коррекциях пересылаются в корневой рэнк, который их и печатает на экране.
    \item Версия в \code{ZEUS} отличается от реализации в \code{zFlare}.
  \end{itemize}\end{block}

  \begin{block}{Обмены}\begin{itemize}
    \item Согласно рекомендациям стандарта, вызов \code{MPI\_Irecv} должен предшествовать вызову \code{MPI\_Isend}. Такой порядок устранил проблемы с обменами при использовании \code{OpenMPI-4}. Для сравнения, та же задача, 20 итераций:
    \txt/OpenMPI-1 = 127 c vs. OpenMPI-4 = 122 с/
  \end{itemize}\end{block}
\end{frame}


\begin{frame}[fragile]{Стиль кода}
  \begin{block}{Рекомендации}\begin{itemize}
    \item Следите за используемыми средствами, в частности, не используйте \code{std::deque} там, где нужен обычный массив.

    \cpp/typedef std::vector<zNeighboursUnit> zNeighboursArray;/
    
    \item Используйте макрос \cppinline/Z_LOC/ вместо \cppinline/__FILE__,__LINE__/.

    \begin{cppcode}
if (!fp)
  throw z::zOpenFileErr (Z_LOC, filename);
    \end{cppcode}

    \item Используйте макрос \cppinline/Z_OVERRIDE/ и модификатор \cppinline/const/.

    \cpp/double kinetic_energy (const PVec& P) const Z_OVERRIDE;/
    
    \item Используйте концепцию RAII, где это возможно.
    
    \cpp/FileHandle fd (file, "wb");/

    \item По возможности, разделяйте код на логические действия, помещая каждое в отдельную функцию.

    \item Объявляйте переменные по месту использования. Локальность улучшает читабельность и помогает компилятору оптимизировать код.
  \end{itemize}\end{block}
\end{frame}



%%==============================================================
\begin{frame}{Масштабируемость}
  \begin{center}
    Компоновка WBLR, $\approx$50 млн. ячеек, кластер в г.\,Саров
  \end{center}
  \begin{figure}
    \includegraphics[width=0.8\textwidth]{scalability_wblr_lam_c20000.pdf}
  \end{figure}
\end{frame}


\begin{frame}{Влияние декомпозиции}
  \begin{figure}
    \includegraphics<1>[width=0.95\textwidth]{wblr_lam_ini.pdf}
    \includegraphics<2>[width=0.95\textwidth]{wblr_lam_r480.pdf}
  \end{figure}
\end{frame}


%%==============================================================
\begin{frame}[fragile]{Декомпозиция: \texttt{decomp.py}}
  \begin{consolecode}
$ python decomp.py -h
usage: decomp.py [-h] [-c N] -r N [-s {auto,equal,slice}] [-v] [-d]
                 {boco,field,graph,mesh,split,view} ...
  \end{consolecode}

  \begin{block}{Управляющие параметры}\begin{itemize}
    \item \code{-r N } "--- желаемое количество рэнков
  \end{itemize}\end{block}

  \begin{block}{Тонкая настройка}\begin{itemize}
    \item \code{-с N } "--- максимальное желаемое количество ячеек в блоках
    \item \code{-s \{auto,equal,slice\} } "--- алгоритм разбиения блоков
  \end{itemize}\end{block}

  \begin{block}{Команды}\begin{itemize}
    \item \code{view} / \code{graph} "--- анализ / с визуализацией, используя \code{Matplotlib}
    \item \code{boco} "--- декомпозиция только \code{*.zboco} (старый файл \code{*\_orig.zboco})
    \item \code{field} "--- декомпозиция только полей
    \item \code{mesh} "--- равносильно \code{boco} \& декомпозиция сетки
    \item \code{split} "--- равносильно \code{field} \& \code{mesh}
  \end{itemize}\end{block}
\end{frame}


\begin{frame}{Алгоритм разбиения по умолчанию: \texttt{auto}}
  \begin{block}{При ограничении $\text{max}_\text{cells}$ для каждого блока размером $I\times J\times K$:}\begin{enumerate}
    \item Вычисляем количество подблоков $n = \left\lceil \dfrac{I\cdot J\cdot K}{\text{max}_\text{cells}} \right\rceil$.
    \item Генерируем тройки "--- $i,j,k$ частей в соответств. направлениях:

    \begin{tabbing}
      \ttfamily for $i = 1$ to $n$ \\
      \qquad\= $n_j = \left\lceil \dfrac{n}{i} \right\rceil$\\[5pt]
      \>\ttfamily for $j = 1$ to $n_j$ \\[5pt]
      \> \qquad\= $k = \left\lceil \dfrac{n}{i\cdot j} \right\rceil$ \\
      \>       \> yield $(i, j, k) \quad \rightarrow \quad \{(i,j,k): n \leqslant i\cdot j\cdot k < 2n\}$
    \end{tabbing}
    
    \item Вычисляем площади разрезов:
    \[
       \forall i,j,k:\quad S_\text{cut}(i,j,k) = (i-1)\cdot J K + (j-1)\cdot I K + (k-1)\cdot I J.
    \]
    
    \item Выбираем тройку с минимальным $S_\text{cut}$.
  \end{enumerate}\end{block}
\end{frame}


\begin{frame}{Анализ: \texttt{decomp.py -r 1024 graph -b 50 z.zboco}}
  \begin{figure}
    \includegraphics[width=0.95\textwidth]{wblr_lam_r1024.pdf}
  \end{figure}
\end{frame}


\begin{frame}{Более тонкая настройка: ключик \texttt{-c}}
  \begin{figure}
    \includegraphics<1>[width=0.95\textwidth]{wblr_lam_r1024_c30000.pdf}
    \includegraphics<2>[width=0.95\textwidth]{wblr_lam_r1024_c20000.pdf}
  \end{figure}
\end{frame}


\begin{frame}{Иной путь}
  \begin{figure}
    \includegraphics[width=0.95\textwidth]{wblr_lam_r1100_c30000.pdf}
  \end{figure}
\end{frame}


\begin{frame}[fragile]{Декомпозиция и восстановление}
  \begin{itemize}
    \item поля отдельно
    \begin{consolecode}
$ python decomp.py ... field -h
usage: decomp.py field [-h] [-n N] proj_file boco_file
    \end{consolecode}

    \item сетка отдельно
    \begin{consolecode}
$ python decomp.py ... mesh -h
usage: decomp.py mesh [-h] [-n N] boco_file
    \end{consolecode}

    \item всё вместе и сразу
    \begin{consolecode}
$ python decomp.py ... split -h
usage: decomp.py split [-h] [-n N] proj_file boco_file
    \end{consolecode}

    \item ведётся лог-файл

    \item восстановление полей
    \begin{consolecode}
$ python recomp.py -h
usage: recomp.py [-h] [-n N] [-v] [-d] proj_file boco_file
    \end{consolecode}
  \end{itemize}

  \begin{block}{Проблемы}\begin{itemize}
    \item Многопоточная версия реализована простым способом, используя \code{Pool} из \code{multiprocessing} и \alert{не работает} в Windows. \alert{Не <<убивается>>} в Linux привычным \code{Ctrl+C}.

    \console`$ pkill -09 decomp.py`
  \end{itemize}\end{block}
\end{frame}



%%==============================================================
\begin{frame}{Вместо заключения}
  Необходимо накопить опыт использования, чтобы исправить недочёты. Приветствуются критические замечания и полезные функциональные пожелания.
  
  \begin{block}{Дальнейшие планы:}\begin{itemize}
    \item Декомпозиция и восстановление для патчей.
    \item Учёт весов для дробного шага.
    \item \ldots
  \end{itemize}\end{block}
\end{frame}



%%==============================================================
\begin{frame}[standout]
  \begin{center}
    \huge Давайте жить дружно!
    
    \large Спасибо за внимание!
  \end{center}
\end{frame}

\end{document}
